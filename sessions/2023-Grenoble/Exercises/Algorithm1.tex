\documentclass[a4paper]{article}
% Setup margins
\usepackage{geometry}
\geometry{
    textwidth=0.76\paperwidth,
    textheight=0.76\paperheight
}

% Math packages
\usepackage{bm}
\usepackage{amsfonts}
\usepackage{bbm}
\usepackage{mathtools}

% Algorithm package with some options
\usepackage[ruled,vlined,linesnumbered]{algorithm2e}

% Let's you split fractions nicely in text
\usepackage{xfrac}

% Can be useful to make space efficient lists
\newenvironment{packed_enum}{
\begin{itemize}
  \setlength{\itemsep}{1pt}
  \setlength{\parskip}{0pt}
  \setlength{\parsep}{0pt}
  \setlength{\topsep}{0pt}
}{\end{itemize}}

\begin{document}

\begin{algorithm}[ht]
\DontPrintSemicolon
Initialize the variables and the level sets\\
\For{each new item $j$} {
    Let $i^*$ be the predicted buyer of item $j$, formally, $pred(j) = i^*$\\
    \While{there exists $i \in S_{j}$ who spends less than $\eta$ fraction of its budget}{
    Continuously allocate fractions of $j$ to buyers on the lowest levels \tcp*{Stage 1}
    }
    \textnormal{ \\}
    \Repeat{either: \begin{packed_enum}
        \item $(1-\eta)$ fraction has been allocated to $i^{*}$
        \item or $j$ is completely sold
        \item or $i^{*}$ exhausted its budget
    \end{packed_enum}}
    {Continuously allocate fractions of $j$ to $i^{*}$ \tcp*{Stage 2}}
    \While{there exists $i \in S_{j}$ with non-exhausted budget and $j$ is not completely sold} {
    Continuously allocate fractions of $j$ to buyers on the lowest levels \tcp*{Stage 3}
    }
}
\caption{Algorithm with Prediction for the Bounded Allocation Problem.}
\label{algo}
\end{algorithm}

\end{document}
