\documentclass[a4paper]{article}
% Setup margins
\usepackage{geometry}
\geometry{
    textwidth=0.76\paperwidth,
    textheight=0.76\paperheight
}

% Frame a block in the text
\usepackage{mdframed}

% Math packages
\usepackage{bm}
\usepackage{amsfonts}
\usepackage{bbm}
\usepackage{mathtools}
\usepackage{dsfont}
\usepackage{amsmath}
\usepackage{amssymb}

% Setup the style for your theorems
\newenvironment{proof}{\noindent\emph{Proof\ }}{\hspace*{\fill}$\Box$\medskip}
\newenvironment{claimproof}{\noindent\emph{Proof of claim\ }}{\hspace*{\fill}$\Box$\medskip}
\newenvironment{plainproof}{\noindent\emph{Proof\ }}{}
\newtheorem{theorem}{Theorem}
\newtheorem{definition}{Definition}
\newtheorem{lemma}{Lemma}
\newtheorem{claim}{Claim}
\newtheorem{proposition}{Proposition}
\newtheorem{corollary}{Corollary}

% Define some math commands
\newcommand{\vect}[1]{\ensuremath{\mathbf{#1}}}
\newcommand{\pred}{\texttt{pred}}
\newcommand{\one}{\ensuremath{\mathds{1}}}

% Let's you split fractions nicely in text (sfrac in text)
\usepackage{xfrac}

% Give compactenum
\usepackage{paralist}

% Gives the spacing at the notation section
\usepackage{colortbl}


\begin{document}

\begin{figure}
\begin{mdframed}
    \begin{align*}
		&& \min \sum_{i=1}^{n} c_{i} x_{i}^{T} &= \min \sum_{i=1}^{n} c_{i} \sum_{t=1}^{T}\bigl( x_{i}^{t} - x_{i}^{t-1}\bigr)\\
    \text{subject to} &&
		\sum_{k=1}^{K} w_{k}^{t} &= 1  && \forall\ t \\
		%
		&& x_{i}^{t} &\geq \sum_{k=1}^{K} w_{k}^{t} s_{i,k}^{t} && \forall\ i, t\\
		%
		&& x_{i}^{t} &\geq x_{i}^{t-1} && \forall\ i, t\\
		%
		&& w_{k}^{t} &\geq 0  && \forall\ t, k
    \end{align*}
    where $1 \leq t \leq T$ and $1 \leq i \leq n$.
    \vspace{5pt}
\end{mdframed}
\caption{Formulation of the \texttt{LIN-COMB} benchmark}
\label{fig:benchmark}
\end{figure}

\clearpage

\begin{theorem} \label{covering-theorem}
The algorithm's cost is at most $O(\ln(K \rho))$-competitive in the \texttt{LIN-COMB} benchmark.
\end{theorem}
%
\begin{proof} Lemma 1 proved that our algorithm creates feasible solutions for the dual problem of the \texttt{LIN-COMB} benchmark relaxation and for the original covering problem. We show that the algorithm's solution increases the primal objective value of the original covering problem by at most $O(\ln(K \rho))$ times the value of the dual solution, which serves as the lower bound on the \texttt{LIN-COMB} benchmark - the best linear combination of the experts' solutions.
\begin{align}
	 \sum_{i=1}^{n} &c_{i} (x_{i}^{t} - x_{i}^{t-1})
		= \frac{1}{\ln(K\rho)} \sum_{i: x_{i}^{t} > x_{i}^{t-1}} c_{i}(x_{i}^{t} - x_{i}^{t-1}) &&  \notag \\
		%
		&\leq \sum_{i: x_{i}^{t} > x_{i}^{t-1}} c_{i}(x_{i}^{t} + \delta_{i}^{t}) \ln \frac{x_{i}^{t-1} + \delta_{i}^{t}}{x_{i}^{t-1} + \delta_{i}^{t}} \\
		%
		&\leq \sum_{i: x_{i}^{t} > x_{i}^{t-1}} c_{i} (x_{i}^{t} + \delta_{i}^{t}) \ln \frac{x_{i}^{t-1} + \delta_{i}^{t}}{x_{i}^{t-1} + \delta_{i}^{t-1}} \\
		%
		&= \sum_{i: x_{i}^{t} > x_{i}^{t-1}} c_{i} \left[ \left(\sum_{k=1}^{K}  s_{i,k}^{t} w_{i,k}^{t} + \frac{1}{K} \sum_{k=1}^{K} s_{i,k}^{t} \right)
			\ln \left(\frac{ \sum_{k=1}^{K}  s_{i,k}^{t} w_{i,k}^{t} + \delta_{i}^{t}}{x_{i}^{t-1} + \delta_{i}^{t-1}}  \right) \right] \\
%
&\leq \sum_{i: x_{i}^{t} > x_{i}^{t-1}} c_{i} \left[ \left(\sum_{k=1}^{K}  s_{i,k}^{t} w_{i,k}^{t} + \frac{1}{K} \sum_{k=1}^{K} s_{i,k}^{t} \right)
			\ln \left(\frac{ \sum_{k=1}^{K}  s_{i,k}^{t} w_{i,k}^{t} + \delta_{i}^{t}}{\sum_{k=1}^{K}  s_{i,k}^{t-1} w_{i,k}^{t-1} + \delta_{i}^{t-1}}  \right) \right]\\
%
	&= \sum_{i: x_{i}^{t} > x_{i}^{t-1}} \sum_{k=1}^{K} (w_{i,k}^{t} + 1/K) c_{i} s_{i,k}^{t}
				\ln \left(\frac{ \sum_{k=1}^{K} s_{i,k}^{t} w_{i,k}^{t}  + \delta_{i}^{t}}{\sum_{k=1}^{K}  s_{i,k}^{t-1} w_{i,k}^{t-1}  + \delta_{i}^{t-1}}  \right) \notag \\
%
%\end{align}
%%
%\begin{align}
%
&=  \sum_{i: x_{i}^{t} > x_{i}^{t-1}} \sum_{k=1}^{K} (w_{i,k}^{t} + 1/K) \biggl( a_{i}^{t} \hat{s}_{i,k}^{t} \gamma^t + \lambda_{i}^{t} \biggr) \\
%
%
&\leq \sum_{i=1}^{n} \sum_{k=1}^{K} (w_{i,k}^{t} + 1/K) \biggl( a_{i}^{t} \hat{s}_{i,k}^{t} \gamma^t + \lambda_{i}^{t} \biggr) \notag \\
%
&= \sum_{i=1}^{n} a_{i}^{t} \biggl(\sum_{k=1}^{K} w_{i,k}^{t} \hat{s}_{i,k}^{t} \biggr) \gamma^t + \sum_{i=1}^{n} \bigg( \sum_{k=1}^{K} w_{i,k}^{t} \biggr) \lambda_{i}^{t}
+ \frac{1}{K}  \sum_{k=1}^{K} \biggl( \sum_{i=1}^{n} a_{i}^{t}  \hat{s}_{i,k}^{t}  \biggr) \gamma^t + \frac{1}{K} \sum_{k=1}^{K} \sum_{i=1}^{n} \lambda_{i}^{t} 		\notag \\
%
&= 2 \gamma^{t} + 2\sum_{i=1}^{n} \lambda_{i}^{t} = \ln(K \rho) \alpha^{t}
\end{align}
%
The above corresponding transformations hold since:
\begin{compactenum}[(1)]
	\setcounter{enumi}{1}
	\item follows from the inequality $a - b \leq a \ln(\sfrac{a}{b})$ for all $0 < b \leq a$;
	\item holds since $\delta_{i}^{t} \geq \delta_{i}^{t-1}$ (because $s_{i,k}^{t} \geq s_{i,k}^{t-1}$ for all $i,k,t$);
	\item is valid because $x_{i}^{t} > x_{i}^{t-1}$, so $x_{i}^{t} = \sum_{k=1}^{K}  s_{i,k}^{t} w_{i,k}^{t}$;
	\item is by the design of the algorithm: $x_{i}^{t-1} \geq \sum_{k=1}^{K}  s_{i,k}^{t-1} w_{i,k}^{t-1}$;
	\setcounter{enumi}{5}
	\item since given that $x_{i}^{t} > x_{i}^{t-1} \geq 0$
	(so $\sum_{k=1}^{K}  s_{i,k}^{t} w_{i,k}^{t} = x_{i}^{t} > 0$), the KKT condition applies;
	\item is true due to the complementary slackness conditions
		and that $\sum_{i=1}^{n} a_{i}^{t}  \hat{s}_{i,k}^{t} = 1$.
\end{compactenum}
\end{proof}

\clearpage

\section{Signal Temporal Logic}

\subsection{Notation}

\begin{center}
\begin{tabular}{ m{15em}  m{22em} } 
$V = \{x_1, x_2, ... , x_n\}$ & Current state of the model as a set of variables \\
$\mathbbm{D} = \mathbbm{D}_1 \times \mathbbm{D}_2 \times ... \times \mathbbm{D}_n$ & Domain of $V$ \\
$\mathbbm{T} = \mathbbm{R}_{\geq 0}$ & Time set\\
$\mathbbm{B} = \{-1, +1\}$ & Boolean set \\
\\
$w: \mathbbm{T} \mapsto \mathbbm{D}$ & A signal $w$\\
$M = \{\mu_1, \mu_2, ... , \mu_k\}$ & Booleanizers $\mu_j: \mathbbm{R}^n \mapsto \mathbbm{B}$\\
\\
$\varphi$ & STL formula \\
$(w, t) \models \varphi$ & The signal $w$ satisfies the STL formula $\varphi$ at time $t$\\
$\chi(\mu, w, t)$ & Characteristic function\\
\\
$\mathcal{I} \subset \mathbbm{T}$ & Time interval (open or closed) as $(i_1, i_2)$ and $i_1 < i_2$\\
$t + \mathcal{I} = \{t + t'\ |\ t' \in \mathcal{I}\}$ & Time interval window from a given time $t$\\
\\
$\mathcal{U}_\mathcal{I}$ & Until operator with interval $\mathcal{I}$\\
$\bigcirc_\mathcal{I}$ & Next operator with interval $\mathcal{I}$\\
$\lozenge_\mathcal{I}$ & Eventually operator with interval $\mathcal{I}$\\
$\square_\mathcal{I}$ & Always operator with interval $\mathcal{I}$\\
\end{tabular}
\end{center}

\subsection{Events}

\begin{center}
\begin{equation} 
\textnormal{NextEvent}(z,r_k) = \left\{
\begin{array}{ll}
\infty\ \textnormal{if}\ (z, \delta z)\ \textnormal{is continuous for}\ r > r_k \\
\min\{r > r_k\ |\ (z[r^-], \delta z[r^-]) \ne (z[r^+], \delta z[r^+])\}\ \textnormal{otherwise}
\end{array}
\right.
\end{equation} 
\end{center}
\end{document}
